\documentclass[a5paper,12pt]{article}
\usepackage[cache=true]{minted}
\usepackage{polyglossia}
\setmainlanguage{russian}
\let\cyrillicfonttt\monofamily
\usepackage{comment}
\usepackage{stmaryrd}

\usepackage[ left=1cm
           , right=2cm
           , top=1cm
           , bottom=1.5cm
           ]{geometry}
\usepackage{amssymb,amsmath,amsthm,amsfonts} 

\usepackage{fontspec}
% \DeclareMathSizes{22}{30}{24}{20}

%\usefonttheme{professionalfonts}
\defaultfontfeatures{Ligatures={TeX}}
%\setmainfont[Scale=1.5]{Times New Roman}
%\setmainfont{Latin Modern Roman}
\setmainfont [ Scale=1]{CMU Serif Roman}
\setsansfont[Scale=1]{CMU Sans Serif}

%\setmonofont[ BoldFont=lmmonolt10-bold.otf
%			, ItalicFont=lmmono10-italic.otf
%			, BoldItalicFont=lmmonoproplt10-boldoblique.otf
%			, Scale=1.5
%]{lmmono9-regular.otf}
%\setmonofont[Scale=1.5]{CMU Typewriter Text}
\setmonofont{CMU Typewriter Text}

\usepackage{unicode-math}
\setmathfont{Latin Modern Math}[Scale=1]
\newcommand*{\arr}{\ensuremath{\rightarrow}}

% Doesn't work?
\renewcommand{\epsilon}{\ensuremath{\varepsilon}}
%\renewcommand{\sigma}{\ensuremath{\varsigma}}
\newcommand{\inbr}[1]{\left<{#1}\right>}
\newcommand{\ruleno}[1]{\mbox{[\textsc{#1}]}}
\newcommand{\bigslant}[2]{{\raisebox{.2em}{$#1$}\left/\raisebox{-.2em}{$#2$}\right.}}
\newcommand{\sem}[1]{\llbracket #1 \rrbracket}
\newcommand{\subtypeof}[2]{ #1\ \text{<:}\ #2}
\begin{document}


\section{1}

\section{2}

\section{3}
Дано:

\begin{align*}
  \subtypeof{o}{o}&\\
  v <: v&\\
    v <: o& \\
    f\langle+x\rangle& \\
        f\langle x\rangle <: o& 
\end{align*}

Запрос
\begin{minted}{ocaml}
(subtype a (ia d)) &&& (subtype d (ia a))
\end{minted}

Вообще алгоритм имени мисонижника делает поиск в ширину и переваривает конъюнкты в ширины, причем количество конъюнктов может увеличиваться. Далее, имеющийся фронт конъюнктов можно анализировать различными способами и просекать альфа-эквивалентность какую-нибудь, запускать гомеоморфные вложения и т.п.

\end{document}
